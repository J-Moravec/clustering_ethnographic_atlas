\documentclass[12pt, a4paper]{article}

\usepackage[top=22mm, bottom=24mm, left=34mm, right=25mm, headsep=6mm,
            footskip=8mm, includeheadfoot, headheight=14pt]{geometry}
\usepackage{booktabs}
\usepackage{caption}
\usepackage{multirow}
\usepackage{makecell}
\usepackage{graphics}

\pagestyle{empty}

\begin{document}

\begin{table}
    \scalebox{0.8}{
    \begin{tabular}{lrrrrrr}
    \toprule
     & & \multicolumn{5}{c}{Comparisons} \\
     \cmidrule{3-7}
     & Variables & 2 vs 3 & 5 vs 6 & 4 vs (2, 3) & 1 vs (5, 6) & ((2,3),4) vs (1,(5,6))\\
    \midrule
    Subsistence & 10 & -0.60 & -1.00 & -0.80 & -1.00 & -1.00\\
    Marriage & 10 & -0.80 & -1.00 & -1.00 & -0.80 & -1.00\\
    Descent & 5 & -0.60 & -1.00 & -1.00 & -1.00 & -1.00\\
    Settlement pattern and size & 2 & -1.00 & -1.00 & -1.00 & -1.00 & -1.00\\
    Political organization & 6 & -0.17 & -1.00 & -0.17 & -1.00 & -1.00\\
    Belief and religion & 2 & 1.00 & 0.00 & -1.00 & -1.00 & -1.00\\
    Games & 1 & 1.00 & -1.00 & 1.00 & -1.00 & -1.00\\
    Sex related taboos and traditions & 4 & 0.50 & -0.50 & -0.50 & -0.50 & -1.00\\
    Sex differences & 11 & 0.55 & -0.73 & -0.45 & -1.00 & -1.00\\
    Age and occupational specialization & 2 & -0.50 & -1.00 & -0.50 & -1.00 & -1.00\\
    Class stratification and slavery & 4 & -1.00 & -0.75 & -0.50 & -1.00 & -1.00\\
    Inheritance of property & 4 & -1.00 & -1.00 & -1.00 & -1.00 & -1.00\\
    Gousing & 10 & 0.20 & -0.20 & -0.40 & -1.00 & -1.00\\
    Climate & 2 & 1.00 & 1.00 & 0.00 & 0.00 & 0.00\\
    \midrule
     Mean &  & -0.19 & -0.73 & -0.62 & -0.92 & -0.97\\
    \bottomrule
    \end{tabular}
    }
    \caption[]{Comparison between distribution of variables in explored clusters and bifurcations (divisions along the tree structure). Negative values mean that the distribution in particular variable are different, while positive values suggest that they come from the same distribution. To increase readability, the 71 variables were summarized into 14 categories and average for each category is shown. Overall, distribution differs greatly in each bifurcation, although variables for climate seem to not play any role in any of them. Instead, variables associated with subsistence, marriage, descent, settlement pattern and size, class stratification and slavery and inheritance of property seems to be the most divisive. The big differentiation between matrilocal and patrilocal societies in the clusters 2 and 3 seems to be determined by the least amount of variables and mostly by three variable categories: settlement pattern and size, class stratification and slavery and inheritance of property.}
\end{table}

\end{document}
